<%# some convenience definitions %>
<%# wiki = @wiki %>
<% full = wiki.filter.clone; full.include_all_namespaces %>
<% wiki.filter.namespace=0 %>
<% users = @wiki.users %>
<% pages = @wiki.pages %>
<% revisions = @wiki.revisions %>
<% topdf = (ENV['RB_TOPDF'] || 'pspdf').to_sym %>
\documentclass{scrartcl}

\usepackage[T1]{fontenc}
\usepackage{booktabs}
\usepackage{graphicx}
\usepackage{tabularx}

\title{Mediawiki Report -- <%= wiki.to_s %>
}
\author{mediawikiparser (<\texttt{klaus.stein@uni-bamberg.de}>)}


\begin{document}
\maketitle


\section{Statistics} % (fold)
\label{sec:statistics}

We have <%=@wiki.pages(full).length%> pages with 
<%=@wiki.revisions(full).length%>  revisions from <%=users.length%> 
users (<%=pages.length%> pages with <%=revisions.length%> revisions in 
Namespace 0).

\subsection{Wiki Statistics} % (fold)
\label{sub:wiki_statistics}

\begin{tabular}{>{\bfseries}lrrrrr}\toprule
  &\textbf{avg} &\textbf{stddev} &\textbf{med} &\textbf{min}
  &\textbf{max}\\
\midrule
<%= 
wiki.global_userstats.collect { |a|
  a.collect { |v| 
    if v.kind_of?(String)
      v
    elsif v.integer? 
      '%5i' % v
    elsif v.nan?
      '---'
    else
      '%7.2f' % v
    end
  }.join('&')
}.join('\\\\')
%>      
\\\bottomrule
\end{tabular}

% subsection wiki_statistics (end)

\subsection{User Statistics} % (fold)
\label{sub:user_statistics}

\begin{tabular}{>{\bfseries}llrrrrr}\toprule
\textbf{User} & & \textbf{\#edits} & \textbf{\#pages} &
\textbf{edits/page} & \textbf{\#self edits} & \textbf{\#foreign
  edits}\\
\midrule
<%= 
wiki.userstats.sort_by { |u,| u.name }.collect { |u,values| 
  ([u.name, u.node_id] + values[0..4].collect { |v|
     if v.kind_of?(String)
       v
     elsif v.integer? 
       '%5i' % v
     elsif v.nan?
       '---'
     else
       '%7.2f' % v
     end
   }).join('&')
}.join('\\\\')
%>
\\\bottomrule
\end{tabular}

% subsection user_statistics (end)

% section statistics (end)


%%% Cummulative Distribution Functions: Lorenz, Gini, Pareto %%%

\section{Cummulative Distribution Functions} % (fold)
\label{sec:cummulative_distribution_functions}

% Authors vs. Revisions
<%
ur = users.collect { |u| u.revisions.length }
ur.gp_plot_lorenz(:title => "Lorenz Curve", :xlabel => "authors",
 :ylabel => "revisions", topdf => 'cdfar.pdf', :size => '480,480')
%>
\begin{center}
  \includegraphics[width=\textwidth]{cdfar}
\end{center}

% Authors vs. Pages
<%
up = users.collect { |u| u.pages.length }
up.gp_plot_lorenz(:title => "Lorenz Curve", :xlabel => "authors", :ylabel => "pages", topdf => 'cdfap.pdf', :size => '480,480')
%>
\begin{center}
  \includegraphics[width=\textwidth]{cdfap}
\end{center}

% Revisions vs. Pages
<%
pr = pages.collect { |p| p.revisions.length }
pr.gp_plot_lorenz(:title => "Lorenz Curve", :xlabel => "pages", :ylabel => "revisions", topdf => 'cdfpr.pdf', :size => '480,480')
%>
\begin{center}
  \includegraphics[width=\textwidth]{cdfpr}
\end{center}

% section cummulative_distribution_functions (end)


%%% Set Filter and Time Raster %%%
<%
filter = wiki.filter.clone # clone filter
raster = wiki.timeraster(:zero => :week, :step => :week) # set time raster to weekly spells
%>

\section{Revision History} % (fold)
\label{sec:revision_history}

% Cummulative Revisions per Week (CRPW)
<% 
crpw = raster[1..-2].collect { |e| filter.endtime = e 
	[ e, wiki.revisions(filter).length ]
	}
Gnuplot.new do |gp|
	gp.set('xdata', 'time')
	gp.set('timefmt', '%Y-%m-%d', true)
	gp.set('format x', '%b %y', true)
	gp.add(crpw, :timefmt => '%Y-%m-%d', 
		:with => 'linespoints', 
		:title => 'revisions')
	gp.set('nokey')
	gp.set('xtics rotate')
	gp.set('title "Cummulative Revisions per Week"')
	gp.set('xlabel "month"')
	gp.set('ylabel "cummulative number of revisions"')
	gp.plot(topdf => 'crpw.pdf', :size => '640,480')
end
%>
\begin{center}
  \includegraphics[width=\textwidth]{crpw}
\end{center}

% Revisions per Week (RPW)
<% 
rpw = raster[1..-2].enum_cons(2).collect { |s,e| 
	filter.revision_timespan = (s..e)
	[ e, wiki.revisions(filter).length ]
	}
Gnuplot.new do |gp|
	gp.set('xdata', 'time')
	gp.set('timefmt', '%Y-%m-%d', true)
	gp.set('format x', '%b %y', true)
	gp.add(rpw, :timefmt => '%Y-%m-%d',
		:with => 'lines', 
		:title => 'revisions' )
	gp.set('xtics rotate')
	gp.set('title "Revisions per Week"')
	gp.set('xlabel "month"')
	gp.set('ylabel "number of revisions"')
	gp.fit(:title => 'trend')
	gp.plot(:svg => 'rpw.svg')
	gp.plot(topdf => 'rpw.pdf', :size => '640,480')
end
%>
\begin{center}
  \includegraphics[width=\textwidth]{rpw}
\end{center}

% section revision_history (end)

\section{Author Participation} % (fold)
\label{sec:author_participation}

% Relative Author Participation
<%
# events
events = wiki.users.collect{ |u| 
	[ u.time_of_first_event, u.time_of_last_event ] 
    }.select { |f,l| f && l }
# authors
authors = raster.enum_cons(2).collect { |s,e|
	filter.revision_timespan = (s..e)
	[ e,
    	wiki.coauthorgraph(filter).remove_lonely_nodes.nodes.length /
        events.reject { |f,l| (f > e) || (l < s) }.length.to_f ]
	}[1..-2]
# Gnuplot
Gnuplot.new do |gp|
	gp.set('xdata', 'time')
	gp.set('timefmt', '%Y-%m-%d', true)
	gp.set('format x', '%b %y', true)
	gp.add(authors, :timefmt => '%Y-%m-%d',
		:with => 'lines',
		:title => 'authors')
	gp.set('nokey')
	gp.set('xtics rotate')
	gp.set('title "Author Participation"')
	gp.set('xlabel "month"')
	gp.set('ylabel "percentage of authors"')
	gp.fit(:title => 'trend')
	gp.plot(:svg => 'rap.svg')
	gp.plot(topdf => 'rap.pdf', :size => '640,480')
end
%>
\begin{center}
  \includegraphics[width=\textwidth]{rap}
\end{center}

% section author_participation (end)

\section{Network Visualization} % (fold)
\label{sec:network_visualization}

% Co-Authorship Network
<%
filter = wiki.filter.clone # clone filter
coauthorship_network = wiki.coauthorgraph(filter) { |n| [ "label=\"\"" ] }.remove_self_links.remove_lonely_nodes
coauthorship_network.to_graphviz("coauthorgraph.pdf", "neato", topdf, "outputorder=edgesfirst", "node [ shape=point, style=filled, fillcolor=white ]" ) { |w|  "[ weight=#{w} ]" }
%>
\begin{figure}[htbp]
	\centering
	\includegraphics[width=\textwidth]{coauthorgraph.pdf}
	\caption{Coauthorship Network}
	\label{fig:coauthorship_network}
\end{figure}

Figure~\ref{fig:coauthor_network} shows the coauthorship network in which the nodes are authors, and a link between any two nodes represents authorship of one or more pages. There are <%= coauthorship_network.nodes.length %> nodes connected by <%= coauthorship_network.links.length %> links in the network. The coauthorship network gives an idea of who is working with whom. Single nodes with above average links to others identify authors who work with many others, usually on many pages. Rather than content, these authors provide most of the structure to the wiki. Clusters of nodes identify authors working in close concert with each other. These authors provide most of the content to the wiki.


%%% TO DO %%%
% Top 5 Authors in terms of Revisions, Degree, Betweenness, Closeness
% wiki.coauthorgraph.nodes.sort_by { |n| -n.revisions.length }.collect { |n| [ n.name, n.revisions.length ] }[0..4]
% wiki.coauthorgraph.degrees.collect { |k,v| [ k, v[0] + v[1] ] }.sort_by { |k,v| -v }.collect { |k,v| [ k.name, v ] }[0..4]
% wiki.coauthorgraph.betweenness.sort_by { |k,v| -v }.collect { |k,v| [ k.name, v ] }[0..4]
% wiki.coauthorgraph.closeness.sort_by { |k,v| -v }.collect { |k,v| [ k.name, v ] }[0..4]
%%% TO DO %%%

% Document Network
<%
filter = wiki.filter.clone # clone filter
page_network = wiki.pagegraph(filter) { |n| [ "label=\"\"" ] }.remove_self_links.remove_lonely_nodes
page_network.to_graphviz("pagegraph.pdf", "neato", topdf, "outputorder=edgesfirst", "node [ shape=point, style=filled, fillcolor=white ]" ) { |w|  "[ weight=#{w} ]" }
%>

\begin{figure}[htbp]
	\centering
	\includegraphics[width=\textwidth]{pagegraph.pdf}
	\caption{Page Network}
	\label{fig:page_network}
\end{figure}

Figure~\ref{fig:page_network} shows the page network in which the nodes are pages, and a link between any two nodes represents a hyperlink from one to the other page. There are <%= page_network.nodes.length %> nodes connected by <%= page_network.links.length %> links in the network.  The page network gives an idea of which page links to which other. Single nodes with above average links to others identify pages which serve as portals. These pages provide most of the structure of the wiki. Clusters of nodes identify pages which revolve around a common theme or topic. These pages provide most of the content of the wiki.


%%% TO DO %%%
% Top 10 Pages (Indegree)
% wiki.pagegraph.degrees.collect {|k,v| [ k, v[0] ] }.sort_by { |k,v| -v }.collect { |k,v| [ k.title, k.revisions.length, v ] }[0..9]
%\begin{tabular}{>{\bfseries}llrrrrr}
%	\toprule
%	\textbf{Title} & \textbf{Indegree} & \textbf{Revisions} &
%...

% Top 10 Pages (Outdegree)
% wiki.pagegraph.degrees.collect {|k,v| [ k, v[1] ] }.sort_by { |k,v| -v }.collect { |k,v| [ k.title, k.revisions.length, v ] }[0..9]
%\begin{tabular}{>{\bfseries}llrrrrr}
%	\toprule
%	\textbf{Title} & \textbf{Outdegree} & \textbf{Revisions} &
%...
%%% TO DO %%%

% section network_visualization (end)

\end{document}
